\documentclass[aspectratio=169]{beamer}
\usepackage{beamertheme-custom}
\usepackage{symbols-custom}

% Custom theme and packages
\usepackage{hyperref}
\usepackage{amsmath, amssymb, bm, graphicx}
\usepackage{xcolor,transparent}
\usepackage{listings}
\hypersetup{hidelinks}
\usetikzlibrary{decorations.pathmorphing}

% Configure listings for Python code
\lstset{
    language=Python,
    basicstyle=\ttfamily\small,
    keywordstyle=\color{blue},
    stringstyle=\color{red},
    showstringspaces=false,
    frame=tb,
    rulecolor=\color{palegreen},
    backgroundcolor=\color{paleyellow},
    xleftmargin=2.5mm,
    xrightmargin=2.5mm,
    framexleftmargin=2.5mm,    % internal left padding
    framexrightmargin=2.5mm,   % internal right padding
    framesep=2.5mm,            % vertical padding (top and bottom)
    numbers=none,
    breaklines=true,
    breakatwhitespace=true,
    escapeinside={(*@}{@*)},
    basicstyle=\ttfamily\scriptsize
}

\definecolor{paleyellow}{rgb}{0.98,0.98,0.9}
\definecolor{brickred}{rgb}{0.8,0.2,0.2}
\definecolor{palegreen}{rgb}{0.9,0.98,0.9}
\renewcommand{\emph}[1]{\textcolor{red}{#1}}

\title{Cognitive modeling}
\author{Joachim Vandekerckhove}
\date{Spring 2025}

\begin{document}


\maketitle
\begin{frame}{Course Description}
Topics in quantitative methods used in cognitive sciences research focusing on process models, model building, parameter estimation, and model evaluation. 

Examples drawn from models and methods used in cognitive sciences research with practical examples.
\end{frame}

\begin{frame}{Tools}
\textbf{Tools and Environment}:
\begin{itemize}[<+->]
    \item Docker Desktop
    \item Ubuntu Linux environment
    \item Python programming language
\end{itemize}\pause
Docker containers are safe sandboxes in which you can screw up your code or even your operating system.  Get in the habit of starting, stopping, destroying, and rebuilding them.\pause

You'll need to create a GitHub repository for this class.
\end{frame}


\begin{frame}{Flipped Lectures}
\begin{itemize}[<+->]
\item[]
\textbf{Recorded Lectures}:
\begin{itemize}
    \item Published Monday afternoons
    \item Review before class time
    \item Often in an Australian accent
\end{itemize}
\item[]
\textbf{In-Person Meetings}:
\begin{itemize}
    \item Wednesdays, 4:00 PM--4:50 PM
\end{itemize}
\item[]
\textbf{Virtual Meetings} (Optional):
\begin{itemize}
    \item Fridays, 4:00 PM--4:50 PM
\end{itemize}
\end{itemize}
\end{frame}

\begin{frame}{Assessment}
\textbf{Approximately Weekly Assignments (50\%)}
\begin{itemize}[<+->]
    \item Mostly technical, implementations of models in python, etc.
    \item Hurdle assignments -- you must pass to progress
    \item Team work and helping each other is encouraged
    \item Self graded!  You decide whether you have this hurdle under control
\end{itemize}\pause
\textbf{Project Assignments (50\%)}
\begin{itemize}[<+->]
    \item Done individually
    \item Graded based on quality of code, appropriate modeling choices, accuracy of interpretation
    \item One optional by end of Week 5, one required by end of Week 10
    \item You will need all the skills from the hurdle assignments to complete the project assignments
\end{itemize}
\end{frame}

\begin{frame}{Assessment}
\textbf{Submission Policy}:
\begin{itemize}[<+->]
    \item No late submissions (assignments due at noon on the due date)
    \item GitHub snapshot taken at the deadline
    \item Recommendation: Set an internal deadline a day earlier than the due date
\end{itemize}
\end{frame}

\begin{frame}{Key Philosophies: Collaboration and Figuring Stuff Out}

\textbf{Hurdle assignments and collaboration}

When you're stuck on something (e.g., getting a script to run), you \emph{have to try to solve} the problem all by yourself for an hour, but then when the hour is up you \emph{have to ask for help}. Failure to try wastes other people's time, failure to ask wastes your time.
\pause

Figuring things out on your own involves Googling for solutions or asking an AI and {experimenting and trying things out to see if they work}.
\pause

\emph{The experimental verification is really important! The AI often gets it wrong!}
\pause

Finding partial solutions from unverified sources, implementing them, and then conducting rigorous tests is a highly generalizable coding paradigm.

\end{frame}



\begin{frame}{Academic Dishonesty \& Resources }
\textbf{Academic dishonesty policy}

There is no tolerance for academic dishonesty or fraud. Any form of fraud designed to circumvent course policies will result in a failing grade. The professor makes no judgment calls regarding academic dishonesty. Any academic dishonesty, no matter how small, will be escalated to academic authorities.

\textbf{Resources}
\begin{itemize}
\item
\textbf{Disability Services}: {https://dsc.uci.edu/}{}
\item
\textbf{Academic Dishonesty}: {https://aisc.uci.edu/students/academic-integrity/index.php}{}
\item
\textbf{Copyright Policy}: {http://copyright.universityofcalifornia.edu/use/teaching.html}{}
\end{itemize}
\end{frame}

\begin{frame}{AI Policy}
\begin{itemize}[<+->]
    \item[]
    \textbf{General AI Policy:}
\begin{itemize}[<+->]
    \item Use of tools like Copilot, ChatGPT, and generative AI is permitted.
    \item AI is a productivity enhancer but not a substitute for knowing what you are doing.
\end{itemize}
\item[]\textbf{Rules}:
\begin{enumerate}[<+->]
    \item Students are individually responsible for all submissions.
    \item Acknowledge use of reference works, websites, and AI tools in comments.
\end{enumerate}
\end{itemize}
\end{frame}

\maketitle

\end{document}
